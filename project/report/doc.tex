\documentclass[a4paper, titlepage,12pt]{article}

\usepackage[margin=3.7cm]{geometry}
\usepackage[utf8]{inputenc}
\usepackage[T1]{fontenc}
\usepackage[swedish,english]{babel}
\usepackage{csquotes}
\usepackage[hyphens]{url}
\usepackage{amsmath,amssymb,amsthm, amsfonts}
\usepackage[backend=biber,citestyle=ieee]{biblatex}
\usepackage[yyyymmdd]{datetime}
\usepackage{titlesec} 
\usepackage{graphicx}
\usepackage{listings}

\usepackage{xcolor}
\definecolor{codegreen}{rgb}{0,0.6,0}
\definecolor{codepurple}{rgb}{0.5,0,0.5}
\definecolor{backcolor}{rgb}{0.97,0.97,0.97}

\lstdefinestyle{mystyle}{
	commentstyle=\color{codegreen},
	keywordstyle=\color{magenta},
	numberstyle=\color{gray}\ttfamily\footnotesize,
	backgroundcolor=\color{backcolor},
	basicstyle=\ttfamily\footnotesize,
	stringstyle=\color{codepurple},
	numbers=left,
	tabsize=4
}

\lstset{style=mystyle}


\iffalse
\titleformat{\subsection}[runin]
  {\normalfont\large\bfseries}{\thesubsection}{1em}{}
\titleformat{\subsubsection}[runin]
  {\normalfont\normalsize\bfseries}{\thesubsubsection}{1em}{}
\fi

\title{Performance Analysis and Simulation of Communication Systems: Project A}

\author{Adam Temmel (adte1700) \& Fredrik Sellgren (frse1700)}

\begin{document}
	\maketitle
	\section{PRNG and Random Variables}
		\subsection{Create a function that implements a linear congruential generator (LCG), accepting as input the parameters: seed, m, a,and c.}
			We implemented our LCG as a class in order to keep it somewhat similiar to the STL generators. The code used is as follows:

			\begin{figure}[h!]
				\begin{lstlisting}[language=c++]
class Lcg {
public:
	using seed_t = uint32_t;

	Lcg(seed_t seed, seed_t m = 100, seed_t a = 13, 
		seed_t c = 1) 
		: seed(seed), m(m), a(a), c(c) {}

	seed_t operator()() {
		seed = ((a * seed) + c) % m;
		return seed;
	}

	// These are kept public to make it easier to change 
	// them later in the lab, a more authentic generator
	// would probably keep them private.
	seed_t seed, m, a, c;
};
				\end{lstlisting}
				\caption{LCG Class}
				\label{fig:lcgimpl}
			\end{figure}



		\subsection{Generate 1000 values uniformly distributed in the range [0,1] using your PRNG. For this case use m=100, a=13 c=1and seed =1;}
		\subsection{Compare the distribution of your values with the distribution of values generated using the \\
		\lstinline{UniformRandomVariable()} of ns-3.}
		\subsection{Comment on the difference in the results and propose values of m, a,and cwhich gives you better results.}
		\subsection{What PRNG does ns-3 use?  What method does ns-3 use to generate a normal random variable?}
		\subsection{Using the \lstinline{time} system command of Linux compare the execution time for the generation of the uniform distribution using your function and ns-3 function}
		\begin{figure}[h!]
			\begin{lstlisting}
# our class: 
time ./waf --run scratch/project  2.54s user 0.20s 
system 107% cpu 2.547 total
# their class: 
time ./waf --run scratch/project  2.59s user 0.17s 
system 107% cpu 2.556 total
			\end{lstlisting}
			\caption{The results from timing a program that generated 1000 random numbers using \lstinline{time}}

			As we can see, our implementation is slightly faster (around $0.01$ seconds) than the Ns3 implementation.
		\end{figure}
		\subsection{Write a second function that generatesan exponential distributionwith mean $\beta>0$ from auniform distributiongenerated using the LCG;Choose one of the methods for generating RV covered in the course and motivate your choice with respect to the specific task.}
		\subsection{Compare your exponential distribution with ns-3 \lstinline{ExponentialRandomVariable()} and the theoretical expression of the probability density function.}
\end{document}
